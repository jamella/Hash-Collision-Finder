\documentclass[12pt]{article}

\usepackage{sbc-template}

\usepackage{graphicx,url}

\usepackage[brazil]{babel}   
%\usepackage[latin1]{inputenc}  
\usepackage[utf8]{inputenc}  
% UTF-8 encoding is recommended by ShareLaTex

% Para usar pseudo algoritmos
\usepackage[]{algorithm2e}

\usepackage{amsmath}
\usepackage{amsfonts}

\sloppy

\title{Instructions for Authors of SBC Conferences\\ Papers and Abstracts}

\author{Giovani Ferreira\inst{1}, Rafael Marconi\inst{1} }

\address{CEUB - Centro Universitário de Brasília \\
  Caixa Postal 4488 -- 70.904-970 -- Brasília -- DF -- Brazil
}

\begin{document}
\maketitle

\begin{abstract}
adhaskjdhaksjd \cite{tanenbaum2002distributed}
\end{abstract}

\section{Introdução}

Na era em que computadores e dispositivos eletrônicos tem assumido um papel crucial, tornou-se muito importante manter
a segurança e integridade de informações transmitidas através desses dispositivos. Para alcançar esse objetivo, é
amplamente adotado o uso de funções hash, obtendo a integridade da seguinte maneira: Em algum ponto do tempo é computado
o valor hash correspondente a uma entrada particular. Em outro ponto subsequente, para verificar que o dado de 
entrada não foi alterado, o valor hash é computado novamente usando a entrada disponível e comparado por igualdade 
com o valor original. Aplicações específicas de função hash incluem a proteção contra vírus e distribuição 
de software \cite{menezes1996handbook}.

Esse cenário apresenta uma característica importante. Se um adversário consegue alterar a entrada e mesmo assim obter 
um valor do hash igual ao valor da entrada original, o destinatário entenderá que os dados recebidos são íntegros. Tal 
ataque foi demonstrado em diferentes trabalhos \cite{lenstra2005colliding}, \cite{stevens2007chosen} e
\cite{lenstra2005possibility}, que provam ser possível alterar um certificado X.509, inclusive a chave pública, e obter o 
mesmo valor hash.

Essa busca por colisão, ou seja, dois valores distintos que possuam o mesmo valor hash, é uma ferramenta importante em 
criptoanálise. Uma gama grande de problemas criptoanalíticos tais como computar logaritmos discretos, achar colisões de 
função hash e o ataque por encontro a médio caminho, meet-in-the-middle, podem ser reduzidos ao problema de achar duas 
entradas distintas, \(a\) e \(b\), para uma função \(f\) de forma que \(f(a) = f(b)\) \cite{van1999parallel}. Portanto, 
para garantir a segurança, é necessário que funções hash possuam a propriedade de ser computacionalmente inviável achar 
duas entradas diferentes que produzam o mesmo valor hash.

(Parágrafo para falar que os computadores são potentes e introduzir os clusters raspberry)
Por outro lado, os recursos tecnológicos em constante evolução possibilitam um poder computacional maior. Com 
processadores capazes de realizar milhões de operações por segundo 

Utilizando desse poder computacional de baixo custo e de princípios básicos de probabilidade (paradoxo do aniversário), 
é possível desenvolver um ataque de força bruta que consiga encontrar uma colisão em funções hash. Tal ataque
é denominado ataque do aniversário (Birthday Attack) e consiste, em sua forma mais simples, de escolher um quantidade \(n\) 
de mensagens aleatórias, computar seus valores hash \(h_1, h_2, ... , h_n\) e testar por colisão entre todos, ou seja, 
\(h_i = h_j\) para \(i, j \in {1, 2, ... , n}\).

Apesar de técnicas para prevenir ataques baseados no paradoxo do aniversário já terem sido apresentadas \cite{aiello1996foiling},
é interessante discutir o ataque do aniversário já que é possível observar como simples ideias matemáticas em conjunto
com um cluster de microprocessadores de baixo custo podem ser aplicadas e usadas como forma de ataque cibernético.

Dessa forma, este artigo consiste de cinco seções. A segunda tratará da explicação e definição de conceitos relacionados, 
como função hash e colisão de valores hash. A terceira consistirá em detalhar e abordar a implementação do cluster de 
raspberry e dos testes serial, paralelo e distribuído. A quarta seção irá expor os resultados obtidos a partir dos testes
e a comparação de performance entre eles. E, por fim,  a quinta seção dissertará sobre as conclusões obtidas através dos testes.

\iffalse
As formas mais comuns de uso de função hash na criptografia são com assinaturas digitais e para
integridade dos dados. Na assinatura digital, uma mensagem longa geralmente passa pelo processo
de hash (usando uma função hash publicamente disponível) e somente o resultado do hash é assinado.
A parte destinatária da mensagem aplica o processo de hash na mensagem, verifica que a assinatura
recebida é correta para esse valor de hash. Isso salva tanto tempo quanto espaço comparado com 
assinar a mensagem diretamente, o que iria tipicamente envolver dividir a mensagem em blocos de
tamanho apropriado e assinar cada bloco individualmente. Note aqui que a inabilidade de achar duas
mensagens com o mesmo valor de hash é um necessidade de segurança, visto que de outra maneira, a
assinatura em um valor hash de uma mensagem seria o mesmo que o de outra, permitindo que o assinante
assine uma mensagem e em um outro ponto no tempo, reivindicar que ele assinou outra \cite{menezes1996handbook}.

The most common cryptographic uses of hash functions are with digital signatures and
for data integrity.  With digital signatures, a long message is usually hashed (using a publicly 
available hash function) and only the hash-value is signed.  The party receiving the
message then hashes the received message, and verifies that the received signature is correct 
for this hash-value. This saves both time and space compared to signing the message
directly, which would typically involve splitting the message into appropriate-sized blocks
and signing each block individually. Note here that the inability to find two messages with
the same hash-value is a security requirement, since otherwise, the signature on one mes-
sage hash-value would be the same as that on another, allowing a signer to sign one message
and at a later point in time claim to have signed another \cite{menezes1996handbook}.
Hash functions may be used for data integrity as follows. The hash-value corresponding to a particular 
input is computed at some point in time. The integrity of this hash-value is protected in some manner. 
At a subsequent point in time, to verify that the input data has not been altered, the hash-value is 
recomputed using the input at hand, and compared for equality with the original hash-value. Specific 
applications include virus protection and software distribution \cite{menezes1996handbook}.
\fi

\section{Conceitos Relacionados} 

\subsection{Função Hash}

Uma função hash é uma função computacionalmente eficiente \(f\) que mapeia uma sequência binária de tamanho 
arbitrário \(\{0, 1\}^n\) para uma sequência binária de tamanho fixo \(\{0, 1\}^x\), chamada de valor hash, hash 
ou digest \cite{menezes1996handbook}. Ou seja, \(f : \{0, 1\}^n \to \{0, 1\}^x\) para \(n \gg x\) e \(x\) igual a 
um valor fixo como \(128, 160, 256\) e \(512\).

Tipicamente, funções hash são construídas a partir de uma função \(h: B \times R \to R\), que recebe um bloco 
de tamanho fixo de uma mensagem junto com um valor hash intermediário e produz um novo valor hash intermediário. 
Uma mensagem recebida \(m \in \{0, 1\}^n\) é tipicamente completada para ter um comprimento que seja múltiplo do 
tamanho do bloco. Em sequência é quebrada em blocos \(m_1, m_2, ... , m_k \in B\) e começando com alguma constante 
\(r_0 \in \mathbb{R}\), a sequência \(r_i = h(m_i, r_{i-1})\) é computada para \(i = 1, 2, ... , k\) e \(r_k\) é 
o resultado do hash para a mensagem \(m\) \cite{van1999parallel}.

\subsection{Colisão Hash}

Funções hash são projetadas para receber uma mensagem de tamanho arbitrário e mapear ela para uma saída de 
tamanho definido. Se o conjunto de entrada \(\mathbb{M}\) é muito maior que o conjunto de saída, \(\mathbb{M}
\gg \mathbb{R}\), então existem valores \(m \in \mathbb{M}\) que irão ser mapeados para um mesmo valor em 
\(\mathbb{R}\). Uma colisão acontece quando encontra-se dois valores \(m_1\) e \(m_2\) sendo que \(m_1 \neq m_2\) 
mas \(h(m_1) = h(m_2)\).

Uma função hash \(h\) é chamada de livre de colisão (\(collision\ free\)) ou resistente a colisão, se ela mapeia 
mensagens de qualquer tamanho para strings de tamanho definido, mas é computacionalmente inviável achar \(x, y\ |\ h(x) 
= h(y)\) \cite{damgaard1989design} \cite{menezes1996handbook}.

\subsection{Paradoxo do Aniversário}

O paradoxo do aniversário é o princípio contraintuitivo que para grupos de somente \(23\) pessoas, existe a 
chance de aproximadamente \(50\%\) de encontrar duas pessoas com o mesmo aniversário (Assumindo que
todos os aniversários são igualmente prováveis e desconsiderado anos bissexto). Comparado a probabilidade
de encontrar alguém nesse grupo com o mesmo aniversário que o seu, aonde se têm \(23\) chances independentes
e portanto uma probabilidade de sucesso de \(\frac{23}{365} \approx 0.06\), esse princípio é baseado
no fato de que existe \(\frac{23 * 22}{2} = 253\) pares distintos de pessoas. Isso leva a uma probabilidade
de sucesso por volta de \(0.5\) (note que isso não é igual a \(\frac{253}{365} \approx 0.7\) já que esses
pares não são independentemente distribuídos) \cite{stevens2012attacks}.

Uma conclusão importante obtida disso é que se uma função hash gera como saída \(n\) bits, sempre haverá um ataque que
roda em tempo \(2^\frac{n}{2}\). Por exemplo, se a saída é de \(128\) bits, então uma colisão pode ser encontrada em tempo
\(\approx 2^{64}\), o que não é considerado suficientemente seguro. Outra observação interessante é que aniversários na 
verdade não são uniformemente distribuídos, mas o paradoxo do aniversário foi montado com a suposição de que eles são 
uniformemente distribuídos.

\subsection{Ataque do Aniversário}

O ataque do aniversário é a aplicação do paradoxo do aniversário em um cenário real, buscando utilizar desse princípio
como uma forma ofensiva de obter, burlar ou alterar informações restritas. No contexto deste artigo, o ataque do aniversário
consistirá na aplicação do princípio para encontrar duas mensagens diferentes que possuam o mesmo valor hash. Sabendo que
o conjunto das \(tags\) é igual a \(\{0,1\}^n\) e dada a função hash \(h : \mathbb{M} \to \{0,1\}^n\), sendo que \(|\ tags\ | 
\approx 2^n\) bits e que \(| \mathbb{M} | \gg 2^n\), o algoritmo genérico para o ataque do aniversario é composto pelas seguintes 
etapas:
\begin{enumerate}
\item Escolhe-se \(2^\frac{n}{2}\) mensagens aleatórias em \(\mathbb{M}\), de forma que \(m_1, m_2, ... , m_{2^\frac{n}{2}}
\in \mathbb{M}\);
\item Para \(i = 1,2, ... , 2^\frac{n}{2}\) computa-se \(t_i = h(m_i)\), aonde \(t_i\) é o valor hash no conjunto das \(tags\),
ou seja, \(t_i \in tags\); 
\item Busca-se por qualquer colisão, isto é, \(t_i = t_j\) para \(i, j \in {1, 2, ... , 2^\frac{n}{2}}\). Caso não seja 
encontrada, volta-se a etapa \(1\) e repete-se com uma amostra diferente de mensagens.
\end{enumerate}

\iffalse
\subsection{Se pa - Distributed System}
A distributed system is a collection of independent computers that appears to its users as a single 
coherent system \cite{tanenbaum2002distributed}.
\fi

\iffalse
\subsection{Message Integrity}

Message integrity is normally maintained via the protocol called “MAC" or Message Authentication Code. 
To briefly explain its mechanism,

In the digital world, the techniques for transmission and integrity assurance of messages are in constant
evolution, aiming more efficient and secure protocols. A protocol's security can be threatened by inumerous
different ways, some possibly unknown yet, and requires a reasonable time of study, tests and attacks
to be considered for use in real applications.

O protocolo de autenticacao HMAC pode ser entendido em detalhes em \cite{krawczyk1997hmac}, mas como uma breve
introducao, consideremos o seguinte exemplo: Alice deseja se comunicar com Bob, a integridade deve ser
mantida, ou seja, the message should not be tampered with or changed to contain false or modified information:
\begin{enumerate}
\item Alice gera uma signing tag \(S(k, m)\), for \(m\) = message and \(k\) = secret key between Alice and Bob
\item Bob ao receber a mensagem, runs a verification algorithm, defined by \(V(k, m, tag) = V(k, m, S(k, m))\)
\item Using the same key, the algorithm will return \(yes\) which shows that message integrity has been mantained
\end{enumerate}

In that way, Bob is able to identify that the message sent by Alice was not compromised and its content
is exactly what Alice had written.

\fi

\section{Implementação do Ataque do Aniversário}

\subsection{Estruturação do Cluster Raspberry}

\subsection{Implementação do Ataque}

A implementação do ataque foi efetuada de seis maneiras diferentes. Consistindo de duas implementações seriais, duas 
paralelas e duas distribuídas. As implementações obrigatoriamente tem que realizar três tarefas diferentes:
\begin{itemize}
 \item Coletar mensagem aleatória \(m \in \mathbb{M}\);
 \item Computar o valor hash \(h_m\) da mensagem;
 \item Comparar o valor hash com os outros valores computados.
\end{itemize}

Elas diferem entre si pela forma como executam essas tarefas. Sendo ou modular, aonde as tarefas são executadas 
separadamente, ou contínua, aonde um pedaço de cada tarefa é executado repetidamente até todas as tarefas serem 
concluídas.

\subsubsection{Implementação Serial Contínua}

A implementação serial foi planejada para executar em um único core de um processador. Sendo feita sequencialmente
e sem distribuição do processamento. O algoritmo serial contínuo funciona da seguinte maneira:
\begin{algorithm}[H]
  n = número de mensagens aleatórias\;
  \While{Primeiro iterador for menor que n}{
    Coleta uma mensagem aleatória\;
    Computa o valor hash correspondente a mensagem\;
    \While{Segundo iterador for menor que o número de hashs já computados}{
      Compara o hash atual com os outros já computados\;
    }
  }
\end{algorithm}

\subsubsection{Implementação Serial Modular}

A implementação serial modular possui as mesmas características da serial contínua, com exceção da sequência em que as
tarefas ocorrem. Elas são executadas separadamente:
\begin{algorithm}[H]
  n = número de mensagens aleatórias\;
  \While{Iterador \(<\) n}{
    Coleta uma mensagem aleatória\;
    Computa o valor hash correspondente a mensagem\;
  }
  \While{Iterador \(<\) número de hashs já computados}{
    Compara o hash atual com os outros\;
    \If{Hash atual \(=\) Outro hash}{
      Colisão encontrada\;
      Aborta a execução\;
    }
  }
\end{algorithm}

\subsubsection{Implementação Paralela Contínua}

\subsubsection{Implementação Paralela Modular}

\subsubsection{Implementação Distribuída Contínua}

\subsubsection{Implementação Distribuída Modular}

\section{Experimentos e Resultados}

\section{Conclusões e Trabalhos Futuros}

\bibliographystyle{sbc}
\bibliography{hash-collision}

\end{document}