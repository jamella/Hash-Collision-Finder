\documentclass[12pt]{article}

\usepackage{sbc-template}

\usepackage{graphicx,url}

\usepackage[brazil]{babel}   
%\usepackage[latin1]{inputenc}  
\usepackage[utf8]{inputenc}  
% UTF-8 encoding is recommended by ShareLaTex

\usepackage{amsmath}
\usepackage{amsfonts}

\sloppy

\title{Instructions for Authors of SBC Conferences\\ Papers and Abstracts}

\author{Giovani Ferreira\inst{1}, Rafael Marconi\inst{1} }

\address{CEUB - Centro Universitário de Brasília \\
  Caixa Postal 4488 -- 70.904-970 -- Brasília -- DF -- Brazil
}

\begin{document}
\[\[\maketitle

\begin{abstract}
adhaskjdhaksjd \cite{tanenbaum2002distributed}
\end{abstract}

\section{Introduction}

Apesar de técnicas para evitar o ataque do aniversario ja terem sido criadas \cite{aiello1996foiling}.

Busca por colisão é uma ferramenta importante em criptoanálise. Uma gama grande de problemas criptoanalíticos
tais como computar logaritmos discretos, achar colisões de função hash e o ataque por encontro a médio caminho,
meet-in-the-middle, pode ser reduzido ao problema de achar duas entradas distintas, \(a\) e \(b\), para uma
função \(f\) de forma que \(f(a) = f(b)\) \cite{van1999parallel}.

%Collision search is an important tool in cryptanalysis. A broad range of cryptanalytic problems
%such as computing discrete logarithms, finding hash function collisions, and meet-in-the-middle
%attacks can be reduced to the problem of finding two distinct inputs, \(a\) and \(b\), to a 
%function \(f\) such that \(f(a) = f(b)\) \cite{van1999parallel}.

As formas mais comuns de uso de função hash na criptografia são com assinaturas digitais e para
integridade dos dados. Na assinatura digital, uma mensagem longa geralmente passa pelo processo
de hash (usando uma função hash publicamente disponível) e somente o resultado do hash é assinado.
A parte destinatária da mensagem aplica o processo de hash na mensagem, verifica que a assinatura
recebida é correta para esse valor de hash. Isso salva tanto tempo quanto espaço comparado com 
assinar a mensagem diretamente, o que iria tipicamente envolver dividir a mensagem em blocos de
tamanho apropriado e assinar cada bloco individualmente. Note aqui que a inabilidade de achar duas
mensagens com o mesmo valor de hash é um necessidade de segurança, visto que de outra maneira, a
assinatura em um valor hash de uma mensagem seria o mesmo que o de outra, permitindo que o assinante
assine uma mensagem e em um outro ponto no tempo, reivindicar que ele assinou outra \cite{menezes1996handbook}.
Funções hash podem ser usadas para manter a integridade da seguinte maneira. O valor do hash correspondente
a uma entrada particular é computado em algum ponto do tempo. A integridade desse valor hash é protegido
de alguma maneira. Em um ponto subsequente no tempo, para verificar que o dado de entrada não foi
alterado, o valor do hash é computado novamente usando a entrada disponível e comparado por igualdade
com o valor do hash original. Aplicações específicas incluem proteção contra vírus e distribuição de
software \cite{menezes1996handbook}.

\iffalse
The most common cryptographic uses of hash functions are with digital signatures and
for data integrity.  With digital signatures, a long message is usually hashed (using a publicly 
available hash function) and only the hash-value is signed.  The party receiving the
message then hashes the received message, and verifies that the received signature is correct 
for this hash-value. This saves both time and space compared to signing the message
directly, which would typically involve splitting the message into appropriate-sized blocks
and signing each block individually. Note here that the inability to find two messages with
the same hash-value is a security requirement, since otherwise, the signature on one mes-
sage hash-value would be the same as that on another, allowing a signer to sign one message
and at a later point in time claim to have signed another \cite{menezes1996handbook}.
Hash functions may be used for data integrity as follows. The hash-value corresponding to a particular 
input is computed at some point in time. The integrity of this hash-value is protected in some manner. 
At a subsequent point in time, to verify that the input data has not been altered, the hash-value is 
recomputed using the input at hand, and compared for equality with the original hash-value. Specific 
applications include virus protection and software distribution \cite{menezes1996handbook}.
\fi

\section{Related Concepts} 

\iffalse
\subsection{Message Integrity}

Message integrity is normally maintained via the protocol called “MAC" or Message Authentication Code. 
To briefly explain its mechanism,

In the digital world, the techniques for transmission and integrity assurance of messages are in constant
evolution, aiming more efficient and secure protocols. A protocol's security can be threatened by inumerous
different ways, some possibly unknown yet, and requires a reasonable time of study, tests and attacks
to be considered for use in real applications.

O protocolo de autenticacao HMAC pode ser entendido em detalhes em \cite{krawczyk1997hmac}, mas como uma breve
introducao, consideremos o seguinte exemplo: Alice deseja se comunicar com Bob, a integridade deve ser
mantida, ou seja, the message should not be tampered with or changed to contain false or modified information:
\begin{enumerate}
\item Alice gera uma signing tag \(S(k, m)\), for \(m\) = message and \(k\) = secret key between Alice and Bob
\item Bob ao receber a mensagem, runs a verification algorithm, defined by \(V(k, m, tag) = V(k, m, S(k, m))\)
\item Using the same key, the algorithm will return \(yes\) which shows that message integrity has been mantained
\end{enumerate}

In that way, Bob is able to identify that the message sent by Alice was not compromised and its content
is exactly what Alice had written.

\fi

\subsection{Hash Functions}

Uma função hash é a função computacionalmente eficiente que mapeia uma sequência binária de tamanho 
arbitrário para uma sequência binária de tamanho fixo, chamada de hash-value, hash ou digest \cite{menezes1996handbook}. 

%A hash function is a computationally efficient function mapping binary strings of arbitrary length 
%to binary strings of some fixed length, called hash-values \cite{menezes1996handbook}.

\subsection{Hash Collision}

Collision Resistance - It is computationally infeasible to find any two distinct input \(x, x'\) 
which hash to the same output, i.e., such that \(h(x) = h(x')\). (Note that here there is free 
choice of both inputs.) \cite{menezes1996handbook}

A hash function \(h\) is called \(collision free\), if it maps messages of any length to strings of
some fixed length, but such that finding \(x\), \(y\) with \(h(x) = h(y)\) is a hard problem. Note 
that we are concentrating here on publicly computable hash functions, i.e. functions that are not
controlled by a secret key \cite{damgaard1989design}.

Hash functions are designed to take a message of arbitrary bitlength and map it to a fixed size
output called a hash result. Let \(H : M \to R\) be such a hash function. Typically, 
hash functions are constructed from a function \(h: B \times R \to R\) which takes a fixed size block 
of message bits together with an intermediate hash result and produces a new intermediate hash result. 
A given message \(m \in \mathbb{M}\) is typically padded to a multiple of the block size and split 
into blocks \(m_1, m_2, ... , m_l \in B\). The padding often includes a field which indicates the 
number of bits in the original message. Beginning with some constant \(r_0 \in \mathbb{R}\), the sequence 
\(r_i = h(m_i, r_{i-1})\) is computed for \(i = 1, 2, ... , l\), and \(r_l\) is the hash result for message 
\(m\) \cite{van1999parallel}.

\subsection{Birthday Paradox}

O paradoxo do aniversário é o princípio contraintuitivo que para grupos de somente 23 pessoas, existe a 
chance de aproximadamente 50\% de encontrar duas pessoas com o mesmo aniversário (Assumindo que
todos os aniversários são igualmente prováveis e desconsiderado anos bissexto). Comparado a probabilidade
de encontrar alguém nesse grupo com o mesmo aniversário que o seu, aonde se têm \(23\) chances independentes
e portanto uma probabilidade de sucesso de \(\frac{23}{365} \approx 0.06\), esse princípio é baseado
no fato de que existe \(\frac{23 * 22}{2} = 253\) pares distintos de pessoas. Isso leva a uma probabilidade
de sucesso por volta de \(0.5\) (note que isso não é igual a \(\frac{253}{365} \approx 0.7\) já que esses
pares não são independentemente distribuídos) \cite{stevens2012attacks}.

\iffalse
The birthday paradox is the counter-intuitive principle that for
groups of as few as \(23\) persons there is already a chance of about one half of finding two 
persons with the same birthday (assuming all birthdays are equally likely and disregarding 
leap years). Compared to finding someone in this group with your birthday where you have 
\(23\) independent chances and thus a success probability of \(\frac{23}{365} \approx 0.06\), this principle is 
based on the fact that there are \(\frac{23 * 22}{2} = 253\) distinct pairs of persons. This leads to 
a success probability of about \(0.5\) (note that this does not equal \(\frac{253}{365} \approx 0.7\) since these 
pairs are not independently distributed) \cite{stevens2012attacks}.
\fi

\subsection{Birthday Attack}

O algoritmo genérico para o ataque do aniversario é composto pelas seguintes etapas:
\begin{enumerate}
\item Dada a função hash \(H : M \to \{0,1\}^n\) e sabendo que o tamanho do conjunto das \(tags\) é \(\approx 2^n\) bits e que \(| M | \gg 2^n\)
\item Escolhe-se \(2^\frac{n}{2}\) mensagens aleatórias em \(\mathbb{M}\), de forma que \(m_1, m_2, ... , m_{2^\frac{n}{2}} \in \mathbb{M}\).
\item Para \(i = 1,2, ... , 2\frac{n}{2}\) computa-se \(t_i = H(m_i)\), aonde \(t_i\) é o hash no conjunto das \(tags\) % is the hash value in the tag space.
\item Busca-se por qualquer colisão, ou seja, \(t_i = t_j\) para \(i, j \in {1, 2, ... , 2^\frac{n}{2}}\). Caso não seja encontrada, volta-se a etapa 1 e 
repete-se com uma amostra diferente de mensagens.
\end{enumerate}

\subsection{Se pa - Distributed System}
A distributed system is a collection of independent computers that appears to its users as a single 
coherent system \cite{tanenbaum2002distributed}.
 
\section{Experiments and Evaluation}

Foram aplicados tecnicas de paralelismo (openmp) e distribuicao (mpi) visando uma mlehora na performance
da busca por colisao. A funcao hash usada nos testes foi a MD5.

\section{Conclusions and Future work}

\bibliographystyle{sbc}
\bibliography{hash-collision}

\]\]
\end{document}